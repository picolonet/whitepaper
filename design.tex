\section{Overall Design}

The Picolo database network is a collection of Picolo Nodes or Cluster of nodes.

A Node represents an instance of a participant in the network. A single machine
can host multiple nodes. It is also possible for a node to span multiple
machines such as in a datacenter, we call such a setup a Cluster-node. It is also possible
for a small cluster of machines to collectively represent two or more nodes.  The details of how
multi-machine nodes are managed are discussed later in Section XXX. For all practical purposes of
understanding the protocols, all nodes (single m/c or multi-m/c) will behave as independent participants with
a certain amount of resources at their disposal (disk space, memory, compute capacity, network connectivity, bandwidth and availability).

    \begin{itemize}
        \item Overall Picolo architecture.
        	Bottom up architecture - network sub system, database sub system. Verifiable logs. Data poisoning detection.
        \item Token economics.
        	Nodes need to put up a stake. Nodes earn tokens paid by dapps or developers who use the network. Who pays for the bandwidth.Options are mining or push the cost to developers
    \end{itemize}


