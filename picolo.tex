\documentclass[preprint,10pt]{elsarticle}
\usepackage{etoolbox}
\makeatletter
\patchcmd{\ps@pprintTitle}{\footnotesize\itshape
       Preprint submitted to \ifx\@journal\@empty Elsevier
       \else\@journal\fi\hfill\today}{\relax}{}{}
\makeatother

\usepackage[margin=2.5cm]{geometry}

\newcommand{\fscale}[1]{#1\linewidth}
\newcommand{\figref}[1]{Fig.~\ref{#1}}
\graphicspath{{./fig/}}

\usepackage{sectsty}
\sectionfont{\Large}
\subsectionfont{\large}
%% Use the option review to obtain double line spacing
%% \documentclass[preprint,review,12pt]{elsarticle}

%% Use the options 1p,twocolumn; 3p; 3p,twocolumn; 5p; or 5p,twocolumn
%% for a journal layout:
%% \documentclass[final,1p,times]{elsarticle}
%% \documentclass[final,1p,times,twocolumn]{elsarticle}
%% \documentclass[final,3p,times]{elsarticle}
%% \documentclass[final,3p,times,twocolumn]{elsarticle}
%% \documentclass[final,5p,times]{elsarticle}
%% \documentclass[final,5p,times,twocolumn]{elsarticle}

%% The graphicx package provides the includegraphics command.
\usepackage{graphicx}
%% The amssymb package provides various useful mathematical symbols
\usepackage{amssymb}
%% The amsthm package provides extended theorem environments
%% \usepackage{amsthm}

%% The lineno packages adds line numbers. Start line numbering with
%% \begin{linenumbers}, end it with \end{linenumbers}. Or switch it on
%% for the whole article with \linenumbers after \end{frontmatter}.
\usepackage{lineno}
\usepackage{url}
\bibliographystyle{unsrt}

%% natbib.sty is loaded by default. However, natbib options can be
%% provided with \biboptions{...} command. Following options are
%% valid:

%%   round  -  round parentheses are used (default)
%%   square -  square brackets are used   [option]
%%   curly  -  curly braces are used      {option}
%%   angle  -  angle brackets are used    <option>
%%   semicolon  -  multiple citations separated by semi-colon
%%   colon  - same as semicolon, an earlier confusion
%%   comma  -  separated by comma
%%   numbers-  selects numerical citations
%%   super  -  numerical citations as superscripts
%%   sort   -  sorts multiple citations according to order in ref. list
%%   sort&compress   -  like sort, but also compresses numerical citations
%%   compress - compresses without sorting
%%
%% \biboptions{comma,round}

% \biboptions{}

\begin{document}

\begin{frontmatter}

%% Title, authors and addresses

\title{Picolo: Fast p2p open database network\tnoteref{title1}}
\tnotetext[title1]{Work in progress, Version 0.1 Draft.}

\author{Picolo Labs\corref{auth1}}
\cortext[auth1]{https://picolo.network}
\address{San Francisco, California}

%% use the tnoteref command within \title for footnotes;
%% use the tnotetext command for the associated footnote;
%% use the fnref command within \author or \address for footnotes;
%% use the fntext command for the associated footnote;
%% use the corref command within \author for corresponding author footnotes;
%% use the cortext command for the associated footnote;
%% use the ead command for the email address,
%% and the form \ead[url] for the home page:
%%
%% \title{Title\tnoteref{label1}}
%% \tnotetext[label1]{}
%% \author{Name\corref{cor1}\fnref{label2}}
%% \ead{email address}
%% \ead[url]{home page}
%% \fntext[label2]{}
%% \cortext[cor1]{}
%% \address{Address\fnref{label3}}
%% \fntext[label3]{}


%% use optional labels to link authors explicitly to addresses:
%% \author[label1,label2]{<author name>}
%% \address[label1]{<address>}
%% \address[label2]{<address>}

\begin{abstract}
%% Text of abstract
Picolo is a fast, scalable, verifiable, fully decentralized, globally distributed transaction oriented database network for
blockchain based applications.  It has a generic SQL interface, provides synchronous replication and external consistency - the strongest form of consistency a database can achieve. It uses a probabilistic replication framework on top of DHTs to achieve a O(1) lookup latency for most queries. It provides high availability in the face of failures by sharding and replicating data across multiple machines. It provides fine grained access control to data by using encryption as the enforcement mechanism and a declarative language to define access rules. 
\end{abstract}

\begin{keyword}
	decentralization \sep blockchain \sep ethereum \sep database \sep sql \sep access control \sep secret sharing \sep threshold cryptography \sep p2p
\end{keyword}

\end{frontmatter}
%-----------------------------------------------------------------------------
%  INTRODUCTION
%-----------------------------------------------------------------------------
\section{Introduction}\label{Sect:Introduction}
We are in the midst of building a new Internet. Blockchain networks like Ethereum have popularized the idea of unstoppable, owner less applications that are run on an open network of untrusted but incentivized nodes without the oversight of a central authority. An increasing number of these decentralized applications (dapps) are being created everyday with evolving data storage needs. The first generation of these dapps stored their data entirely on a blockchain itself while the current generation are storing it on decentralized file storage systems like IPFS with just hashes of these data stored on the blockchain. Although this method works for dapps with simple storage needs like the need for storing data as a blob, it is very limiting for dapps with more complex requirements such as being able to store data at a more fine grained level and querying it efficiently. A popular option is then to use a cloud hosted database (ex: Google’s Cloud SQL) but that turns a dapp into a non-dapp by introducing centrality into the system. \newline\newline
In this paper we introduce Picolo, an open database network that combines the benefits of a traditional database and peer-to-peer software. It is the world’s first NewSQL decentralized database system that offers features such as a SQL interface, distributed transactions, external consistency \cite{External_Consistency}, lock-free reads and snapshot reads. It is an open network that allows anyone with spare computing power and disk space to join the network and get rewarded for hosting and serving structured data.
\newline\newline
Features that set Picolo apart from other decentralized databases:
\begin{itemize}
	\item Transactions can be applied across rows, columns and tables across nodes
	\item Client controlled replication and data placement
	\item Supports storage of typed data
	\item Supports semi-relational structure for tables
	\item Configurable backups and restore mechanisms
	\item Allows for verifiable transaction logs
	\item Detects data poisoning
	\item Provides distributed query processing
	\item Enables fine grained access control to data
	\newline
\end{itemize}
Rest of the paper is structured as follows: in section 2, we discuss some related work. Section 3 presents the overall design of the system, section 4 presents the network subsystem, section 5 presents the database subsystem and section 6 presents our approach in dealing with problems that arise in a network made up of untrusted nodes and discusses an incentive mechanism to make them behave as per system's needs. In section 7, we propose a message exchange protocol in the same vein as http but for relational data sharing. Section 8 concludes the paper.

%-----------------------------------------------------------------------------
%  BACKGROUND SECTION
%-----------------------------------------------------------------------------
%-----------------------------------------------------------------------------
%  BACKGROUND SECTION
%-----------------------------------------------------------------------------
\section{Background and Related Work}
There have been attempts in the academia at building peer-to-peer data management systems (PDMS). PeerDB \cite{PeerDB} pioneered a full-fledged data management system that supports fine-grain content-based searching on a distributed network of nodes with heterogeneous schemas. It proposed a "code goes to data" paradigm where mobile agents are employed to perform query processing at a peer node in order to reduce network bandwidth. PIER \cite{PIER} provides a relational data model and query operators on top of any distributed storage system. It embraces the notion of \textit{data independence} and only concerns itself with query processing. It does not provide any sort of replication or ACID guarantees of a database and does not maintain any indexes of its own. Piazza \cite{Piazza} describes a system where peers have pairwise schema mappings between heterogeneous schemas and how these can be transitively extended to answer queries from peers separated by more than one edge. Queries are reformulated at each peer to match the local schema before execution. While this approach works well for a few reformulations, a large number of them may result in information loss or returning of irrelevant results.
\newline\newline

    \begin{itemize}
        \item FileSystems or file-level semantics: IPFS, FileCoin. Storj.
        \item Bluezelle, fluence, pepperdb
        \item Blockchains that offer faster transactions.
        \item Storing data on the cloud. AWS. GCE.
    \end{itemize}




%-----------------------------------------------------------------------------
%  OVERALL DESIGN SECTION
%-----------------------------------------------------------------------------
\section{Overall Design}

The Picolo database network is a collection of Picolo Nodes or Cluster of nodes.

A Node represents an instance of a participant in the network. A single machine
can host multiple nodes. It is also possible for a node to span multiple
machines such as in a datacenter, we call such a setup a Cluster-node. It is also possible
for a small cluster of machines to collectively represent two or more nodes.  The details of how
multi-machine nodes are managed are discussed later in Section XXX. For all practical purposes of
understanding the protocols, all nodes (single m/c or multi-m/c) will behave as independent participants with
a certain amount of resources at their disposal (disk space, memory, compute capacity, network connectivity, bandwidth and availability).

    \begin{itemize}
        \item Overall Picolo architecture.
        	Bottom up architecture - network sub system, database sub system. Verifiable logs. Data poisoning detection.
        \item Token economics.
        	Nodes need to put up a stake. Nodes earn tokens paid by dapps or developers who use the network. Who pays for the bandwidth.Options are mining or push the cost to developers
    \end{itemize}




%-----------------------------------------------------------------------------
%  NETWORK SUB-SYSTEM SECTION
%-----------------------------------------------------------------------------
\section{Network Subsystem} 

The network layer is a fully decentralized, peer-peer overlay routing layer among the nodes participating in the Picolo
database network. The most important goal of the network layer is to help locate content as efficiently and quickly as
possible while surviving certain kinds of failure or malicious intent. Peer-peer networking literature refers to this
functionality as Decentralized Object Location and Routing (DOLR) \cite{dolr2003}. The network layer focuses on routing
messages, such as database queries, read/write requests or other management functions to the respective nodes which can
satisfy them. 

The overlay layer can be implemented on top of any datagram network protocol such as UDP or IP. Specifically on the Internet, we realize its implemented on IP or IPv6 protocols.
The peer-to-peer overlay routing infrastructure offers efficient, scalable, location-independent
routing of messages directly to nearby copies of an object or service using only localized resources [TP].

-- Self-repairing.
-- Soft-state based routing.

\subsection{Background}

In this Section, we provide a brief overview of peer-peer systems both in the research and product community that have
influenced the literature over the last 20 years.

Napster \cite{Napster} was one of the first popular service that provided much of the original inspiration for peer-peer
systems although its database was centralized.  DNS is an example of a widely deployed distributed and largely
decentralized key-value database that powers every lookup and interaction on the Internet \cite{Mockapetris_1988}. DNS
relies on special root servers to bootstrap the lookup protocol. The Freenet \cite{freenet_thesis, Clarke_2001} and the
Gnutella \cite{Gnutella} p2p systems were popular in the previous decade for file sharing. Both systems were designed
for sharing of large files over a longer duration of time. Content reliabilty including lookup reliability and network
latency goals were necessary in this enviroment. 

The second generation of peer-peer systems include research driven projects such as Chord \cite{Stoica_2001}, Content
Addressable Network (CAN)
\cite{Ratnasamy_2001}, Pastry \cite{Rowstron_2001}, Tapestry \cite{tapestry2004} and
Kademlia. Chord along with CAN, Tapestry and Pastry developed the concept of distributed hash tables (DHTs) as a
fundamental mechanism for content-based addressing. They built over the scalability and self-organizing properties of
both FreeNet and Gnutella by providing a definite answer to a lookup query in a bounded number of network hops. In fact,
these protocols are able to locate content within \( O(log N) \) where \(N\) is the number of nodes in the system. From
an 
API perspective, these overlays provide a key-based routing (KBR) interface that supports deterministic routing of
messages to a live node that has the responsiblity for the "value" corresponding to the given key. These systems also
support high level APIs such as Dynamic Object Location and Routing (DOLR) \cite{dolr2003}. Most DHT's use the concept
of consistent hashing to distribute the load evenly among the nodes.

{\em Consistent hashing:}
Typical hashing based schemes do a good job of spreading load through a known, fixed collection of servers. Since the
Blockchain consists of nodes on the Internet which can appear and disappear based on incentives and other criteria, our
assumption is that machines come and go as they crash or are brought into the network. Also,
the information about what nodes are functional propagates slowly through the
network, so that clients may have incompatible “views” of which nodes are available to replicate data. (Note that a
node can also be a client). This makes standard hashing useless since it relies on clients agreeing on which nodes are responsible for serving a particular
page.

Like most hashing schemes, consistent hashing assigns a set of items to buckets so that
each bin receives roughly the same number of items.  Unlike standard hashing schemes, a small change in the bucket set
does not induce a total remapping of items to buckets. In addition, hashing items into slightly different sets of
buckets gives only slightly different assignments of items to buckets. 

Chord uses such hash functions to map nodes and content uniformly to a circular 160-bit
namespace. Chord improves the scalability of consistent hashing by removing the requirement that every node knows about
every other node. Each node only maintains about \(O (log N) \) state information about other nodes in an \( N \) node
network. When nodes join or leave, they require \( O(log^2 N) \) messages to keep the network updated.

CAN routes messages in a {\em d}-dimensional space where each node maintains a routing table with \(O(d)\) entries and
any node can be reached in \(O(dN^{1/d})\) routing hops. CAN's routing table does not grow with network size, but the
number of routing hops grows faster than \(log N\).

When compared to Chord and CAN, Pastry and Tapestry take network distances into account when constructing overlay
topologies. While Chord and CAN use shortest overlay hops and other runtime heuristics, both Tapestry and Pastry
construct locally optimal routing tables to reduce any routing inefficiencies. 

Pastry and Tapestry share some similarities to the work by Plaxton et al \cite{Plaxton_1997} and to the routing layer in
the Landmark hierarchy \cite{Tsuchiya_1988}. The approach consists of routing based on address prefixes or otherwise
called prefix-based routing. However, both Pastry and Tapestry include an ability to self-organize the network structure
and achieve network locality in content mapping which also lends support for replication.
In addition, Tapestry also allows some application-based locality management by "publishing" location pointers
throughout the network for efficiently locating content and services.

Kademlia \cite{kademlia} is another p2p DHT-based routing system that uses prefix-based routing by arranging 160-bit IDs
(node IDs and content IDs) in a binary tree style data-structure for efficient routing. It uses an XOR-based distance
metric for building the routing table and for the routing algorithm itself. In terms of its performance and other
features, it is very similar to the above systems such as Chord, CAN, Pastry and Tapestry, but it offers simplicity in
its routing and lookup algorithms which make it attrative for implementation. IPFS \cite{ipfs} uses a version of
Kademlia for locating files for decentralized applications.

Other notable systems include Viceroy \cite{viceroy} which provides logarithmic hops through nodes with constant degree
routing tables. SkipNet \cite{skipnet} uses a multidimensional skip-list data structure to support overlay routing,
maintaining both a DNS-based namesapce for operational locality and a randomized namespace for network locality. Other
overlay proposals such as Koorde \cite{koorde} and Naor et al \cite{simple_hash} attain lower bounds on local routing
state but oversimplify some of the other features. 

The third generation of P2P research includes building applications on top of these DHT systems, validating them as
novel infrastructures or tuning them for specific use cases. For example, applications such as PAST \cite{past} and
SCRIBE \cite{scribe} are built on top of Pastry. Decentralized file storage application project OceanStore \cite{oceanstore} was built
on top of Tapestry, while CFS \cite{cfs} was build on top of Chord. FarSite \cite{farsite} uses a conventional
distributed directory service and could be built on top of Pastry. Another example of an overlay network is the Overcast
System \cite{overcast}, which provides reliable single-source multicast streams.

\subsection{Design of the Picolo overlay network}

The Picolo overlay network consists of a p2p DHT based lookup for mapping key to objects, content or services. There is a caching layer that allows for frequenly used items to be propagated closer to the demand endpoints in the network including ability to replicate as needed.

\subsubsection{Picolo namespace} 
1. Nodes and content map to a single 256-bit space. We can use existing hash functions such as SHA-256 for this purpose.

This space can be segregaged using a namespace identifier, which allows multiple such "naming layers" to co-exist. For example, one way to assign naming layers could be based on a per-application type.

For the rest of this section, the discussion gets isolated to within a single naming layer.

\subsubsection{Core API}
Picolo's network layer supports an API similar to a standard p2p overlay network, for a detailed discussion refer
\cite{dolr2003}. The primary goal of the API is to publish and locate objects or service identifiers within a given
namespace. All operations that occur in a namespace can be replicated across namespaces as needed.

Currenlty, we support the following API methods in a decentralized manner:
\begin{itemize}
    \item Publish()
    \item Unpublish()
    \item Lookup()
    \item RemoteCall()
\end{itemize}

\subsubsection{Routing and Lookup}
Explain how a DHT works. Add a layer that takes care of data locality

\subsection{Node Dynamics}

Failures, node departures, node additions.

\subsection{Replication and caching}
Beehive stuff for O(1) lookups for power law queries

Structured peer-peer distributed hash tables can implement lookups in \( O(log N)\) hops. Since each hop could add between 100-500
ms of latency, for a network of 10K to 100K nodes this computes to about 4-5 seconds for a lookup (or more). Thus, we
require a caching and replication strategy to reduce this cost especially for popular items. The discussion in this
Section applies to any content or service thats is made available through the p2p system and thus extends beyond the
Database as a blockchain service provided by Picolo.

There has been significant prior work on caching and replicating strategies for peer-peer overlay networks. Methods such
as Beehive \cite{beehive} provide a closed-form replication algorithm based on a derived equation that guarantees
\(O(1)\) lookup but require full knowledge of the network, such as number of nodes and popularity values for each data
item. While this system might be good for analysis and benchmark purposes, its implementation is not pratical for
Picolo. Kelips \cite{kelips} is a probabilistic algorithm that also provides \(O(1)\) lookup performance (in the
expected case) by dividing the network into \(O(\sqrt(N))\) affinity groups each of \(\sqrt(N)\) size. They use the
gossip protocol to replicate content to all nodes within an affinity group. Work by Gupta et al \cite{one_hop_lookup}
explore the tradeoff between routing table size and lookup latency. They offer a guaranteed \(O(1)\) lookup by
maintaining routes to each and every node in the network. Farsite \cite{farsite} provides a better ttradeoff by using
routing tables of \(O(dn^{1/3})\) size to route within \(O(d)\) hops.

Other peer-peer applications such as PAST \cite{past} and CFS \cite{cfs} use fixed size caches on intermediate nodes to
cache the objects being queried. While they are unable to provide closed-form analytic bounds on query time, their
average case performance is reasonably good.

We draw on the above literature to find the right balance between routing table size, cache and replication storage at
nodes versus lookup latency. The strategy presented in this Section achieves \(O(1)\) lookup on average for networks with standard node
and popularity dynamics. In other words, we allow for node joins, failures, unexpected departures (including malicious
intent) along with changes in service or object popularity. We also allow for "flash crowds", that is, an item, object
or service (such as a table-shard, or certain rows in a table) can quickly gain popularity (as given by standard
Internet virality models \cite{virality_model}).

\begin{itemize}
    \item A node caches or replicates an item with a probability that is proportional to the number of queries it is
        expected to get for that item in the upcoming time interval T.
    \item The difference between a cache and a replica is application specific. For the Database application that the
        network layer hosts, it depends on whether a particular table or a shard is allowed to have write permissions on
        the replica.
    \item When a node caches or replicates an item, it affects the probabilitic demand distribution for the item at
        nodes that are on the routing path which would have gotten the request has this item not been cached.
    \item Thus, by repeating this algorith iteratively, it would converge to the best caching pattern which would reduce
        lookup times with high probability for all items on the network.
\end{itemize}

\subsection{P2P Connectivity protocol}
Use gossip or some other protocol to maintain network topology, connections

\subsection{Cryptoeconomics and Byzantine behavior}
How will malicious nodes affect the system and how to mitigate/prevent/recover. Do nodes have incentive to participate in network discovery

\subsection{Analytics and Debug/Fault diagnosis}



%-----------------------------------------------------------------------------
%  DATABASE SUB-SYSTEM SECTION
%-----------------------------------------------------------------------------
\section{Database Subsystem}
In this section, we discuss core database concepts like byzantine paxos, role of timestamps in achieving external consistency, distributed transactions and sharding. Concepts that are most new to a database network built for the  decentralized world like distributed query processing, dynamic clustering, data sovereignty and decentralized fine-grained access control are presented.

\subsection{Consensus, replication \& sharding} \label{sec:paxos}
All data on Picolo is replicated for durability and high availability. Storage consumers have the option of choosing a replication factor that suits their needs. They can also choose where to locate their data based on where their users are located to achieve better consistency or to comply with regulations like GDPR. This data locality can easily be achieved by instructing the network layer to only allow replicas that belong to a geographic region (\cref{sec:node_location}) to join the replica group. Consensus is achieved amongst replicas by a running a variant of paxos that is tolerant to byzantine faults. We modify the algorithms in \cite{byzantine_paxos} to make the leader election more frequent and add a slashing condition that punishes byzantine behavior. There are two roles that each replica can take: \textit{proposer and acceptor}. Proposers initiate changes to the state by proposing new commands (key-value pairs) to be appended to a \textit{sequence} from which the state is generated and acceptors vote on which sequence to accept. The system moves through different `views'. A view can be thought of as a discrete time period (in the order of minutes) with a monotonically increasing view number $view_{num}$ and has a distinguished proposer called \textit{leader}. If one imagines that each replica has a number from the set ${1..\mathcal{N}}$ where $\mathcal{N}$ is the number of replicas, then the leader for a view can simply be chosen as $leader_{view} = view_{num} \enspace \% \enspace \mathcal{N}$. There are two modes in which the consensus process happens: $fast$ and $classic$.
\newline\newline
\textbf{Fast mode}: In fast mode (equivalently, \textit{leaderless mode}), proposers can directly send commands to acceptors bypassing the leader. This obviates the need for \textit{phase 1b} messages of the classic paxos. In a network where replicas are present in distant geographic locations, the savings could be significant. Note that all messages are digitally signed, so the senders can be uniquely identified. A message is a tuple \textit{($view_{num}$, seq)} where $seq$ consists of a prefix that is the last accepted command sequence suffixed with new commands. This differs from the classic paxos algorithm where only singular values are passed around and offers two major advantages:
\begin{itemize}
	\item Commands need not be exactly similar - commands can appear in differing orders in different replicas as long as they are commutative.
	\item Proposers don't need a promise from acceptors that they will not accept values with a lower \textit{ballot number} 
\end{itemize}
In the context of a database, two commands are commutative if they are mutating independent records that don't depend on each other for state calculation. For e.g reading a row and writing to another row in a table are commutative operations where as reading and writing to the same row may not be. Even writing to different rows if they have different timestamps is non-commutative if \textit{external consistency} is to be maintained (\cref{sec:hybrid_time}). The relaxed definition of command similarity helps replicas achieve consensus quicker compared to the usual case. Since we cannot assume synchronicity of the network, messages may appear in different order at different replicas. So as long as they are commutative, we can tolerate the order difference and proceed with the consensus process. The acceptors always accept a sequence with the highest length, so they don't need to send back the ballot number promise.
\newline\newline
\textbf{Classic mode}: It is possible that the acceptors are unable to make progress in fast mode $i.e$ append new commands to their sequences when they receive concurrent proposals. Since the network is asynchronous and messages may reach out of order, if they are non-commutative and are of the same length, the acceptors cannot agree on the order by themselves. So they fallback to the leader to arbitrage an order for them. They send their sequences to the leader of the current view $leader_{view}$ who then executes a classic ballot to achieve consensus.
\newline\newline
\textbf{Byzantine fault detection}: Once acceptors receive new proposals, they first verify if the new sequence contains as a prefix an already accepted sequence by them in the past. If not, they simply reject it. If it does, then they sign their acceptance and multicast it to other acceptors. Other acceptors then perform the same check and if it passes, signal their acceptance by $appending$ their signature and multicast it again. This process continues until $\mathcal{N} - f$ acceptors each receive messages with $\mathcal{N} - f$ signatures at which point, the sequence is considered agreed upon and a message is sent back to the proposer indicating consensus. During this process if an honest acceptor receives a multicast that contains signatures of an acceptor $\mathcal{M_A}$ on two non-commutative sequences (see \figref{fig:byz_faults}), then it will trigger a slashing condition (\cref{sec:slashing}). Since we require at least $\mathcal{N} - f$ acceptors to agree on a proposal, at least one of them is guaranteed to be honest and it will trigger the slashing condition. 
\begin{figure}[h!] \centering
	\includegraphics[width=\fscale{1}]{byz_faults.png}
	\caption{Byzantine fault detection during Paxos. Replica A sending conflicting messages is detected by B and C}
	\label{fig:byz_faults}
\end{figure}

\subsubsection{Sharding}
Picolo automatically partitions data into multiple shards when it grows too big for any one replica. Each shard consists of a set of key ranges (typically 64MB). A key range is the smallest atomic unit that is replicated aka governed by a paxos group. It is also the smallest unit of movement when data from one replica needs to be sharded into distinct replica sets. The process of sharding is a well studied problem in databases and implementations can be readily found, so we omit a detailed discussion here. 

\subsection{External consistency, distributed transactions \& hybrid time} \label{sec:hybrid_time}
Using atomic clocks, google external ntp service, stratum 1 ntp servers. Use logical time appended to physical time. Handle clock skew. Nodes that drift too apart should die. 
\newline\newline
\textbf{mvcc}: Keys have timestamps and queries can be made in the past or fetched from a snapshot in the past. Client configurable backup and restore mechanisms. How does this affect data sovereignty?

\subsection{Distributed query processing \& Dynamic clustering} \label{sec:dynamic_cluster}
High level architecture of a Picolo node looks like \figref{fig:node_arch}
\begin{figure}[h!] \centering
	\includegraphics[width=\fscale{1}]{node_arch.png}
	\caption{A Picolo node}
	\label{fig:node_arch}
\end{figure}
The metadata depicted in the figure contains schema metadata to be exposed to the world. Nodes that wish to keep data private should not export the data's metadata to the metadata container. Suppose there is a table called \textit{users} with four columns: \textit{username}, \textit{firstname}, \textit{lastname} and \textit{email}. The exported metadata entry might look like:
\begin{center}
	\begin{tabular}{| c | c |} 
		\hline
		Entity & Keywords \\ [0.5ex] 
		\hline
		users & user, people, customer, profile\\ 
		\hline
		name & name \\
		\hline
		firstname & fn, {f\_name}, {first\_name} \\
		\hline
		lastname & ln, {l\_name}, {last\_name} \\
		\hline
		email & id, contact \\ [1ex] 
		\hline
	\end{tabular}
\end{center}
When a query is posed to any node in the system, the node first tries to fulfill it with a local query before passing it along to its neighbors. Remember, nodes in the system host data with heterogeneous schemas. Hence the keywords are used to find semantically similar data. Similarity metrics that determine whether a query matches a heterogeneous schema are not discussed here.
\newline\newline
\textbf{Dynamic clustering}
Semantic proximity metrics are used to find nodes that are hosting semantically similar schemas. Overtime, these nodes are discovered and are clustered together for better query performance by reducing the number of network hops required.

\subsection{Data sovereignty \& Access control} \label{sec:access_control}
Picolo supports two different schema types: \textit{application controlled} and \textit{user controlled}. Applications can use user controlled schemas to put users in absolute control of data and better comply with regulations like GDPR. For example, a decentralized twitter may  want users to have control over their tweets. Users can use any third party client or Picolo's official clients to interact with their tweets effectively rendering the decentralized twitter just another client to the data albeit with better features. \newline\newline
When semantics don't allow to put user in control of data, application controlled schemas can be used. An example here would be a decentralized ticketing application where users should not be given fine grained control to selectively delete data about which tickets they bought.
\newline\newline
\textbf{Access control} 
Applications and users may want to share data with other parties but may wish to impose access controls. There are a few ways of achieving this including building an API on top of the data or using proxy re-encryption techniques. We use a secret sharing scheme instead as it gels well with p2p nature of the system and does not introduce asymmetry like other solutions. Rules can be defined by a SQL like declarative language at any granularity desired like at the level of a single row or a cell. An example row level granularity rule looks like:\newline \newline
\texttt{SELECT  * \newline FROM users \newline WHERE email=foo@bar.com \newline NODE (SELECT nodeId FROM nodes WHERE domain=application)} \newline \newline
An example value level (only username is given access to) granularity rule looks like:\newline \newline
\texttt{ SELECT username \newline FROM users \newline WHERE email=foo@bar.com \newline NODE (SELECT nodeId FROM nodes)}\newline\newline
Here \texttt{NODE} is a new clause we introduce to SQL dialect.

\section{Crypto economics, Mechanism design, Attacks on the system}
\subsection{Verifiable data structures}
Trillian stuff
\subsection{Attacks}
Possible attacks on the system
\subsection{Crypto economics}
How cryptonomics can be leveraged to prevent such attacks
\subsection{Mechanism design}
How is the system constructed to make sure that desired outcome is achieved

In this Section we discuss how Picolo handles failures and malicious nodes at the Database and the Network layer. Our mechanism design consists of these main pieces:
Nodes need to put up a stake before they can join the network
Usage of byzantine paxos
Usage of trillian and writing checkpoints to a separate blockchain (ethereum most likely)
Incentivizing nodes for correct behavior

We aim to devise Casper FFG style slashing conditions based on the detection of violation of conditions of 2 and 3 above.

Assumptions in the Attack model:

We assume a standard Byzantine failure model, i.e., the attacker is able to make changes to the messages at the Network layer on a node or alter the behavior at the Database layer for the node. The attacker is also able to coordinate the behavior of multiple nodes in real-time to achieve a desired attack scenario. We also allow for the adversary to delay communication between honest nodes so long as the adversary has the ability to do so given the topology of the overlay network (i.e. the adversary should be part of the routing path).

We also assume that the attacker is computationally bound, i.e., they cannot gather enough computing resources to subvert state-of-the-art cryptographic techniques such as Elliptic curve, crypto-hash functions (such as SHA-256) or encryption schemes such as AES.

Byzantine behaviour at the Database layer:

The central tenet of our design is to use strong cryptographic primitives to push the Byzantine behavior at the Database layer down to either a DoS style attack vector or to the network layer where we handle it using a combination of crypto-incentives, DHT management and detection.

At the Database level, each node is responsible for the standard CRUD operations for a given table (or a shard). Each such operation is protected using a cryptographic signature of the entity (dapp or user) that authorizes the request. Thus a malicious node is only able to destroy the data and not alter it. For example, every write request has a signature that certifies the request. Thus, a Merkle proof audit and signature on every table can deter any malicious manipulation of the data stored. 

The only attack vectors that remain are various versions of DoS style disruption, where a single node or a collection of nodes disrupt the network by delaying messages or destroying the data stored. 

Network level Byzantine behavior:

Here we discuss how to handle attack vectors where the adversary can drop network messages, delete data or both. 

The routing layer is based on a cryptographic DHT. Thus, it is not possible for the attacker to “choose” to store a particular shard or content. The only way an attacker can do a targeted DoS is by flooding the network with a majority of nodes such that with high probability one of the compromised nodes gets the responsibility for a targeted table or shard. This mapping includes the backup nodes, thus limiting the effectiveness of a targeted attack.
Caching and replication layer incentivizes faster network links and nodes:
The caching and replication layer prioritizes faster network links and nodes to serve requests. Thus, nodes that are dead, unresponsive or slow will get fewer requests over time. Thus, a DoS style attack scenario will cause diminishing impact over time.
While the network adapts to such a disruption, it is possible for the attacker to gain enough stake and temporarily cause a slowdown.
However, their stake (or trust level) would go down and they would be responsible for fewer network functions over time essentially degrading their involvement over time.	
Detection: It is be possible to detect such DoS style behavior quickly at the network layer since the overlay topologies tend to share many links at the layer-3 on the Internet. For example, if a node appears to delay messages at the network level while maintaining fast routes and content availability might indicate malicious intent.
The detection “service” can run on the nodes with the maximum stake or trust. They can detect anomalies between different layers of a potentially malicious node and agree to reduce the trust level of that node.
Such a detection service can only be thwarted if the malicious node coordinates its behavior across all observable metrics such that it appears as if it is naturally faulty. For this case, it is indistinguishable from a genuinely faulty node for all practical purposes which is handled at the DHT layer (node addition, deletion, failure modes).


\section{MX Protocol}
Nodes in the network need to speak the same language for efficient discovery and communication of data. Hence the following message exchange scheme is proposed similar to \cite{Protocol_Spec}. Note that the exact mapping between the following messages and underlying transport protocol is not discussed here and may change depending on the final transport protocol chosen (QUIC vs TCP)
\newline
\newline
\textbf{PUT}:  A PUT message contains the query to be run, an optional list of nodes to run the query against and an optional max number of hops (needed in case of an empty node list). This is used for creating or updating data in the system.
\newline
\newline
\textbf{GET}: A GET message contains the query to be run, max number of hops, the number of results to retrieve, the mode of retrieval (pull vs push) and a transaction identifier. Client sends this to a server to retrieve results that match the query. Parameters in GET can be varied depending on application needs - a streaming application may choose the push mode in which server pushes data to the client as it becomes available up until the specified number is met. A latency sensitive application may choose to retrieve a small number of results in a batch in one pull.
\newline
\newline
\textbf{SEND}: Server responds to each GET message with one or more SEND messages with results. In pull mode, there is only one SEND message followed by an END message where as in push mode there are multiple SEND messages followed by an END message.
\newline
\newline
\textbf{END}: Server sends END message to indicate to the client that it has finished sending all results in response to a particular GET message.
\newline
\newline
\textbf{CLOSE}: A client can send a CLOSE message to the server to indicate that it no longer is interested in the remaining query results and close the transaction. It doesn't have to wait until all the results are retrieved.
\newline
\newline
\textbf{OK}: Server sends this message to a client as a positive acknowledgment to the client's message.
\newline
\newline
\textbf{ERROR}: Server sends this message to a client as a negatively acknowledge to the client's message.

%-----------------------------------------------------------------------------
%  CONCLUSIONS
%-----------------------------------------------------------------------------
\section{Conclusion}
We combine results from P2P networking and database community.

Novel way where data is stored in a p2p network, data access is audited, access controls, encryption, user control of data


%-----------------------------------------------------------------------------
%  BIBLIOGRAPHY
%-----------------------------------------------------------------------------
\section{References}
\bibliography{./bib/picolo.bib}
\end{document}

%% The Appendices part is started with the command \appendix;
%% appendix sections are then done as normal sections
%% \appendix

%% \section{}
%% \label{}

%% References
%%
%% Following citation commands can be used in the body text:
%% Usage of \cite is as follows:
%%   \cite{key}          ==>>  [#]
%%   \cite[chap. 2]{key} ==>>  [#, chap. 2]
%%   \citet{key}         ==>>  Author [#]

%% References with bibTeX database:


%% Authors are advised to submit their bibtex database files. They are
%% requested to list a bibtex style file in the manuscript if they do
%% not want to use model1-num-names.bst.

%% References without bibTeX database:

% \begin{thebibliography}{00}

%% \bibitem must have the following form:
%%   \bibitem{key}...
%%
%% Example figure:
%%
%% \begin{figure}[h]
%% \centering\includegraphics[width=0.4\linewidth]{placeholder.png}
%% \caption{Figure caption}
%% \end{figure}
%% 
% \bibitem{}

% \end{thebibliography}
